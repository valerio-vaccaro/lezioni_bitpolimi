\documentclass[aspectratio=169]{beamer}
\usepackage{multicol}
\usepackage{qrcode}
\usepackage{datetime}
\usepackage{hyperref}
\usepackage{xcolor}
\usepackage{tikz}

\usetheme{Antibes}
\setbeamertemplate{footline}[frame number]{}

\newcommand{\makemycolor}[2]{%
    \pgfmathsetmacro{\hue}{(#1/100)^1.715*0.79}%
    \definecolor{myhsbcolor}{hsb}{\hue,1,1}%
    \textcolor{myhsbcolor}{#2}%
}

\hypersetup{
    colorlinks=true,
    linkcolor=blue,
    filecolor=magenta,      
    urlcolor=blue,
    pdftitle={Il Mining},
    %pdfpagemode=FullScreen,
    }

\urlstyle{same}

\title{Il Mining} 
\author{Valerio Vaccaro}
\newdate{date}{04}{03}{2024}
\date{\displaydate{date}}
\logo{}

\begin{document}

\begin{frame}[plain,noframenumbering]
    \titlepage
    \begin{center}
        \includegraphics[height=1cm]{logo.png}
    \end{center}
\end{frame}

\begin{frame}[noframenumbering]
    \tableofcontents
\end{frame}

\begin{frame}{Meme}
    \begin{center}
        \includegraphics[height=5cm]{missing.jpg}
    \end{center}
    * entro la fine del corso capirete tutti i meme.
\end{frame}

\begin{frame}{Riassunto}
    
\end{frame}

\section{Il Mining - L’economia di Bitcoin e la creazione di valuta}

\section{Costruzione Block Header}

\section{Mining del Blocco}

\section{Cambiando le Regole di Consenso}

\section{Attacchi al Consenso}

\section{Bibliografia}

\begin{frame}{Bibliografia} 
    \begin{itemize}
        \item Andreas M. Antonopoulos, "Mastering Bitcoin", 2015
        \item Adam Back, "Hashcash-a denial of service counter-measure", 2002
        \item Satoshi Nakamoto, "Bitcoin: A Peer-to-Peer Electronic Cash System", 2008
        \item Hal Finney, "Reusable Proofs of Work", 2004
    \end{itemize}
\end{frame}

\begin{frame}{Altre risorse} 
Tra le altre risorse utili mi piace citare:
    \begin{itemize}
        \item Ovviamente \href{https://t.me/BitPolimi}{BitPolimi} che ha organizzato queste lezioni.
        \item \href{http://satoshispritz.it}{Satoshi Spritz} - eventi serali a scadenze regolari per parlare di Bitcoin (a Milano ci incontriamo ogni mercoledì dalle 18).
        \item \href{https://t.me/ventunobtc}{Ventuno} - Podcast, raccolta di libri e materiali su Bitcoin.
        \item \href{http://officinebitcoin.it}{Officine Bitcoin} - lezioni su telegram da 30 minuti per la risoluzione di problemi pratici.
    \end{itemize}
\end{frame}


\begin{frame}{Domande} 
    \begin{center}
        \includegraphics[height=5cm]{domande.jpg}
    \end{center}
\end{frame}

\begin{frame}[plain]
    \begin{center}
        \includegraphics[height=8cm]{fine.jpg}
    \end{center}
\end{frame}

\end{document}
